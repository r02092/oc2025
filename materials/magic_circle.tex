\documentclass[tikz]{standalone}
\usepackage{luacode,luatexja-fontspec,tikz}
\setmainjfont{APOTFReisho101StdN}
\usetikzlibrary{decorations.text}
\begin{luacode*}
	function circletext(text,count)
		for pos,code in utf8.codes(string.rep(text,count)) do
			tex.sprint("{}"..utf8.char(code))
		end
	end
\end{luacode*}
\newcommand{\circletext}[2]{
	\directlua{circletext("#1",#2)}{}%
}
\begin{document}
	\begin{tikzpicture}[even odd rule]
		% 円
		\draw[line width=.4cm](0,0)circle[radius=16.6];
		\draw[line width=.2cm](0,0)circle[radius=16.1];
		\draw[line width=.6cm,rotate=90,postaction={
			decorate,decoration={
				raise=13pt,
				reverse path,
				text along path,
				text={|\fontsize{46pt}{46pt}\selectfont|\circletext{高知工科大学情報学群画像情報工学研究室}{3}||},
				text align={fit to path}
			}
		}](0,0)circle[radius={7.1/sin(30)}];
		% 六芒星
		\draw[line width=.3cm](0,{7.1/sin(30)})--({-7.1/tan(30)},{-7.1})--({7.1/tan(30)},{-7.1})--cycle;
		\draw[line width=.3cm]({-7.1/tan(30)},{7.1})--({7.1/tan(30)},{7.1})--(0,{-7.1/sin(30)})--cycle;
		% 円
		\draw[line width=.3cm](0,0)circle[radius=7.1];
		% 四角形
		\draw[line width=.2cm](0,{5*sqrt(2)})--({5*sqrt(2)},0)--(0,{-5*sqrt(2)})--({-5*sqrt(2)},0)--cycle;
		\draw[line width=.2cm](-5,5)--(5,5)--(5,-5)--(-5,-5)--cycle;
		% 八卦
		\foreach \i/\bin in {6/{0,0,0},5/{0,0,1},4/{1,0,1},3/{0,1,1},2/{1,1,1},1/{1,1,0},0/{0,1,0},-1/{1,0,0}}
			\foreach \j in {4.1,3.5,2.9}{
				\pgfmathparse{array({\bin},(4.1-\j)/.6)}
				\ifodd\pgfmathresult
					\draw[line width=.3cm]({\i*45+atan(-1/\j)}:{sqrt(\j^2+(-1)^2)})--({\i*45+atan(1/\j)}:{sqrt(\j^2+1^2)});
				\else{
					\draw[line width=.3cm]({\i*45+atan(-1/\j)}:{sqrt(\j^2+(-1)^2)})--({\i*45+atan(-.1/\j)}:{sqrt(\j^2+(-.1)^2)});
					\draw[line width=.3cm]({\i*45+atan(.1/\j)}:{sqrt(\j^2+.1^2)})--({\i*45+atan(1/\j)}:{sqrt(\j^2+1^2)});
				}\fi
			}
		% 太極図
		\fill
			(0,0)circle[radius=2.2]
			(0,2)arc(90:-90:1)--(0,0)arc(90:270:1)--(0,-2)arc(270:90:2)--cycle
			(0,1)circle[radius=.25]
			(0,-1)circle[radius=.25];
	\end{tikzpicture}
\end{document}